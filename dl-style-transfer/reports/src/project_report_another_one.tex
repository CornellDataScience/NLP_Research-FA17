\documentclass{article}

\usepackage[english]{babel}
\usepackage[utf8]{inputenc}
\usepackage{graphicx}

\usepackage[margin=1in]{geometry}

\usepackage{amsmath}
\usepackage[colorlinks=true, allcolors=blue]{hyperref}

\title{Neural Style Transfer on Text\\Project Report One}

\author{Yuji Akimoto, Ryan Butler, Cameron Ibrahim, Luca Leeser}

\begin{document}

\maketitle

\section*{Abstract}

The exploratory period of this project is quickly coming to a close, and we're ramping up our 
effort to get definite results. At this time, we are likely going to be using a CNN based
architecture, in collaboration with competitive layers - either integrated into the layers of the CNN themselves,
or added to the end of the network, as in the style of the Fully Connected Nets commonly added to the end of CNN
architectures. Additionally, we are currently putting together a dataset of alternative text sources, to address
our need for more distinct styles for experiment purposes. We will be training an initial CNN architecture later
today.


\section{Introduction}

Our current research has moved into a more experimental stage, as we begin to train the inital models we have created. These experiments will be
use a 10 book dataset of distinct styles. The hope is that this will allow us to more easily determine success and failure. Failure here would 
render our model unviable for use with Yelp


\section{Methods and Results}

We are currentlly in the process of training our initial models to begin experiments. Loss function has shown improvement. Created a 10 book dataset of sentences
for training uses.

\section{Individual Work}

\begin{description}

	\item [Yuji Akimoto]
		Improved an initial CNN based architecture for text representation. Has yet to be trained. 

	\item [Ryan Butler]
		Further research into KATE, Discrete Autoencoders, ResNet, and Inception/Google to search
		for methods we may be interested in incorporating into our model.

	\item [Cameron Ibrahim]
		Assembling the dataset of a number of books pulled from Project Gutenberg. Training Yuji's model. Generated word embeddings for dataset.

	\item [Luca Leeser]
		Further refined the facenet L2 based loss function for use with text.

\end{description}

\section{Further Studies}

In the event of total failure of our models, we will pivot to trying to explain exactly \emph{why} our methods have not worked. This in itself would make up an interesting research topic, and would easily fill the paper.
\section{References}

\begin{thebibliography}{9}
	\bibitem{Neural Style}
	Leon A. Gatys, Alexander S. Ecker, Matthias Bethge.
	\textit{A Neural Algorithm of Artistic Style}.
	CoRR, 2015.

	\bibitem{Style Transfer Summary}
	Yongcheng Jing, Yezhou Yang, Zunlei Feng, Jingwen Ye, Mingli Song.
	\textit{Neural Style Transfer: A Review}.
	CoRR, 2017.

	\bibitem{Generative models}
	Justin Johnson, Alexandre Alahi, Fei-Fei Li.
	\textit{Perceptual Losses for Real-Time Style Transfer and Super-Resolution}.
	Stanford University.

	\bibitem{Markovian GANs}
	Chuan Li, Michael Wand.
	\textit{Precomputed Real-Time Texture Synthesis with Markovian Generative Adversarial Networks}.
	CoRR, 2016.

	\bibitem{Writing Style Conversion}
	Se Wong Jang, Jesik Min, Mark Kwon.
	\textit{Writing Style Conversion using Neural Machine Translation}.
	Stanford University.

	\bibitem{Text style transfer}
	Tianxiao Shen, Tao Lei, Regina Barzilay, Tommi Jaakkola.
	\textit{Style Transfer from Non-Parallel Text by Cross-Alignment}.
	CoRR, 2017.

	\bibitem{Autoencoder}
	Barak Oshri, Nishith Khandwala.
	\textit{There and Back Again: Autoencoders for Textual Reconstruction}.
	Stanford University.

	\bibitem{KATE}
	Yu Chen, Mohammed J. Zaki.
	\textit{KATE: K-Competitive Autoencoder for Text}. Rensselaer Polytechnic Institute.
	
	\bibitem{Shallow-CNN}
	Yoon Kim.
	\textit{Convolutional Neural Networks for Sentence Classification}. CoRR, 2014.

\end{thebibliography}


\end{document}
